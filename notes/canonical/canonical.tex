\documentclass[a4paper,10pt]{article}
%\documentclass[a4paper,10pt]{scrartcl}



\input{../preambles/preamble}
\input{../preambles/unicode}

\setmainlanguage{english}
\setotherlanguages{german,greek,russian}

\input{../preambles/math-single}
\input{../preambles/math-brac}
\input{../preambles/math-thm}
\input{../preambles/phys-chem}

\setromanfont[Mapping=tex-text]{Linux Libertine O}
% \setsansfont[Mapping=tex-text]{DejaVu Sans}
% \setmonofont[Mapping=tex-text]{DejaVu Sans Mono}

\usepackage[style=authoryear-icomp,
			backend=biber]{biblatex}
\addbibresource{singular-dynamics.bib}

\title{Notes on Canonical Singular Dynamics}
\author{Yi-Fan Wang (王\ 一帆)}
%\date{}

\begin{document}
\maketitle

\section{Canonical formalism}

Lagrangian with velocity
\begin{equation}
L^\text{v} \coloneqq \fat{L}{\dot{q} = v}
\end{equation}
Equations of motion
\begin{equation}
\sum_j M_{ij}\dot{v}_j = K^\text{v}_i,\quad
\dot{q}_i = v_i.
\end{equation}
where
\begin{equation}
\rfun{M_{ij}}{q,v} \coloneqq \frpa{^2 L^\text{v}}{v_i\,\partial v_j}.
\end{equation}

Adding
\begin{equation}
p_i \coloneqq \frpa{L^\text{v}}{v_i} \eqqcolon \rfun{\ol{p}_i}{q,v}.
\end{equation}
Variation of
\begin{equation}
\sfun{S}{q, p; v} \coloneqq \int\dd t\,\sbr{L^\text{v} + \sum_i 
p_i\rbr{\dot{q}_i - v_i}}.
\end{equation}
gives the \emph{extended Euler--Lagrange equations}
\begin{equation}
\dot{q}_i = v_i,\quad
\dot{p}_i = \frpa{L^\text{v}}{q_i},\quad
p_i = \frpa{L^\text{v}}{v_i}.
\end{equation}

Extended Hamiltonian
\begin{equation}
\rfun{H^\text{v}}{q, p; v} \coloneqq \sum_i p_i v_i - L^\text{v}.
\end{equation}
Identities
\begin{equation}
\frpa{H^\text{v}}{q_i} \equiv - \frpa{L^\text{v}}{q_i},\quad
\frpa{H^\text{v}}{p_i} \equiv v_i,\quad
\frpa{H^\text{v}}{v_i} \equiv p_i - \frpa{L^\text{v}}{v_i}.
\end{equation}
Variation of
\begin{equation}
\sfun{S}{q, p; v} \coloneqq \int\dd t\,\sbr{\sum_i 
p_i \dot{q}_i - H^\text{v}}
\end{equation}
gives the \emph{extended canonical equations}
\begin{equation}
\dot{q}_i = \sbr{q_i, H^\text{v}}_\text{TP},\quad
\dot{p}_i = \sbr{p_i, H^\text{v}}_\text{TP},\quad
\frpa{H^\text{v}}{v_i} = 0,
\end{equation}
where the total Poison bracket is defined as
\begin{equation}
\sbr{f^\text{v}, g^\text{v}}_\text{TP} \coloneqq 
\sum_i\rbr{\frpa{f^\text{v}}{q_i}\frpa{g^\text{v}}{p_i} -
\frpa{f^\text{v}}{p_i}\frpa{g^\text{v}}{q_i}}.
\end{equation}

$v_a = \rfun{\ol{v}_a}{q,p}$ can be solved, $a = 1, 2, \ldots, r_M$; $v^\alpha$ 
cannot be solved, $\alpha = r_M + 1, \ldots, n$, where $r_M = \rank M$.

(need to show $v_a = \rfun{\ol{v}_a}{q,p_a}$)

\emph{Primary constraints in the standard form}
\begin{equation}
\rfun{\phi_\alpha^{(0)}}{q, p} \coloneqq
\fat{\frpa{H^\text{v}}{v_\alpha}}{\cbr{v_\alpha = \ol{v}_\alpha}} \equiv
p_\alpha - \fat{\frpa{L^\text{v}}{v_\alpha}}{\cbr{v_a = \ol{v}_a}}.
\end{equation}

\emph{Total Hamiltonian}
\begin{equation}
H^\text{t} \coloneqq \fat{H^\text{v}}{\cbr{v_a = \ol{v}_a}} \equiv
\rfun{H^\text{v}}{q, p; \cbr{\rfun{\ol{v}^a}{q, p_a}, v_\alpha}}.
\end{equation}

\emph{Subspace of primary constraints}
\begin{equation}
\Gamma_\text{P} = \cbr{ \rbr{q, p} | \rfun{\phi_\alpha^{0}}{q, p} = 0, \forall 
\alpha}
\end{equation}

Since
\begin{equation}
\frpa{H^\text{t}}{v_\alpha} =
\fat{\frpa{H^\text{v}}{v_\alpha}}{\cbr{v_a = \ol{v}_a}} = \phi_\alpha^{(0)}
\equiv p_\alpha - \fat{\frpa{L^\text{v}}{v_\alpha}}{v_a = \ol{v}_a},
\end{equation}
$H^\text{t}$ is linear in $v_\alpha$. One writes
\begin{equation}
\rfun{H^\text{t}}{q, \cbr{p_a}; \cbr{p_\alpha}, \cbr{v_\alpha}} = \rfun{H}{q, 
\cbr{p_a}} + \sum_\alpha v_\alpha \phi_\alpha^{(0)},
\end{equation}
where $H$ is the \emph{canonical Hamiltonian} or simply \emph{Hamiltonian}. 

need to show $H$ is independent of $\cbr{p_\alpha}$.

Canonical equations with primary constraints
\begin{align}
\dot{q}_a &= \sbr{q_a, H}_\text{TP} + \sum_\beta v_\beta 
\sbr{q_a, \phi_\beta}_\text{TP}, \\
\dot{p}_i &= \sbr{p_i, H}_\text{TP} + \sum_\beta v_\beta 
\sbr{p_i, \phi_\beta}_\text{TP}, \\
\rfun{\phi_\alpha^{(0)}}{q, p} &= 0,
\end{align}
where $v_\beta$'s are undetermined.

Weak equality: $f \approx g$ iff $\fat{f}{\Gamma_\text{P}} = 
\fat{g}{\Gamma_\text{P}}$.





\section{Examples}

%\begin{equation}
%L^\text{v} = \frac{1}{2} \sum_{i,j}\rfun{W_{ij}}{q} v_i v_j + \sum_i 
%\rfun{\eta_i}{q} 
%v_i - \rfun{V}{q}.
%\end{equation}

%\begin{equation}
%p_i = \frpa{L^\text{v}}{v_i} = \sum_{i,j} W_{ij} v_j + \eta_i.
%\end{equation}

%Let
%\begin{align}
%\sum_j W_{ij} e_j^{(a)} &= \lambda^{(a)} e_i \neq 0, \\
%\sum_j W_{ij} e_j^{(\alpha)} &= 0.
%\end{align}

\subsection{Toy systems}

\subsection{Parametrised systems}

\subsubsection{Non-relativistic point particle}

\subsubsection{Relativistic charged point particle}

\subsubsection{Neutral scalar field}

\subsection{Proca action}

\subsection{Dirac field}

\subsection{Gauge theories}

\subsubsection{Electrodynamics}

\subsubsection{Spinor electrodynamics}

\subsubsection{Yang--Mills theory}

\subsubsection{Yang--Mills--Higgs theory}

\subsection{Gravitation theories}

\subsubsection{Einstein--Hilbert action}



\printbibliography

\end{document}
