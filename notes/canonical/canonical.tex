\documentclass[a4paper,11pt]{article}
%\documentclass[a4paper,11pt]{scrartcl}



\input{../preambles/preamble}
\input{../preambles/unicode}

\setmainlanguage{english}
\setotherlanguages{german,greek,russian}

\input{../preambles/math-single}
\input{../preambles/math-brac}
\input{../preambles/math-thm}
\input{../preambles/phys-chem}

\setromanfont[Mapping=tex-text]{Linux Libertine O}
% \setsansfont[Mapping=tex-text]{DejaVu Sans}
% \setmonofont[Mapping=tex-text]{DejaVu Sans Mono}

\usepackage[style=authoryear-icomp,
			backend=biber]{biblatex}
\addbibresource{../singular-dynamics.bib}

\title{Notes on Canonical Singular Dynamics}
\author{Yi-Fan Wang (王\ 一帆)}
%\date{}

\begin{document}
\maketitle

\section{Classical formalism}

Lagrangian with velocity
\begin{equation}
L^\text{v} \coloneqq \fat{L}{\dot{q} = v}
\end{equation}
Equations of motion
\begin{equation}
\sum_j M_{ij}\dot{v}_j = K^\text{v}_i,\quad
\dot{q}_i = v_i.
\end{equation}
where
\begin{equation}
\rfun{M_{ij}}{q,v} \coloneqq \frpa{^2 L^\text{v}}{v_i\,\partial v_j}.
\end{equation}

Adding
\begin{equation}
p_i \coloneqq \frpa{L^\text{v}}{v_i}.% \eqqcolon \rfun{\ol{p}_i}{q,v}.
\end{equation}
Variation of
\begin{equation}
\sfun{S}{q, p; v} \coloneqq \int\dd t\,\sbr{L^\text{v} + \sum_i 
p_i\rbr{\dot{q}_i - v_i}}.
\end{equation}
gives the \emph{extended Euler--Lagrange equations}
\begin{equation}
\dot{q}_i = v_i,\quad
\dot{p}_i = \frpa{L^\text{v}}{q_i},\quad
p_i = \frpa{L^\text{v}}{v_i}.
\end{equation}

Hamiltonian with velocity
\begin{equation}
\rfun{H^\text{v}}{q, p; v} \coloneqq \sum_i p_i v_i - L^\text{v}.
\end{equation}
Identities
\begin{equation}
\frpa{H^\text{v}}{q_i} \equiv - \frpa{L^\text{v}}{q_i},\quad
\frpa{H^\text{v}}{p_i} \equiv v_i,\quad
\frpa{H^\text{v}}{v_i} \equiv p_i - \frpa{L^\text{v}}{v_i}.
\end{equation}
Variation of
\begin{equation}
\sfun{S}{q, p; v} \coloneqq \int\dd t\,\sbr{\sum_i 
p_i \dot{q}_i - H^\text{v}}
\end{equation}
gives the \emph{extended canonical equations}
\begin{equation}
\dot{q}_i = \sbr{q_i, H^\text{v}}_\text{P},\quad
\dot{p}_i = \sbr{p_i, H^\text{v}}_\text{P},\quad
\frpa{H^\text{v}}{v_i} = 0,
\end{equation}
where the \emph{Poisson bracket} is defined as
\begin{equation}
\sbr{f^\text{v}, g^\text{v}}_\text{P} \coloneqq 
\sum_i\rbr{\frpa{f^\text{v}}{q_i}\frpa{g^\text{v}}{p_i} -
\frpa{f^\text{v}}{p_i}\frpa{g^\text{v}}{q_i}}.
\end{equation}

$v_a = \rfun{\ol{v}_a}{q,p;\cbr{v_\alpha}}$ can be solved, $a = 1, 2, \ldots, 
r_M$; $v_\alpha$ 
cannot be solved, $\alpha = r_M + 1, \ldots, n$, where $r_M = \rank M$.

(need to show $v_a = \rfun{\ol{v}_a}{q,p_a}$)

\emph{Primary constraints in the standard form}
\begin{equation}
\rfun{\Phi_\alpha}{q, p} \coloneqq
\fat{\frpa{H^\text{v}}{v_\alpha}}{\cbr{v_\alpha = \ol{v}_\alpha}} \equiv
p_\alpha - \rfun{\ol{p}_\alpha}{q, \cbr{p_a}},
\end{equation}
where
\begin{equation}
\rfun{\ol{p}_\alpha}{q, \cbr{p_a}} 
\coloneqq \fat{\frpa{L^\text{v}}{v_\alpha}}{\cbr{v_a = \ol{v}_a}}.
\end{equation}


\emph{Total Hamiltonian}
\begin{equation}
H^\text{t} \coloneqq \fat{H^\text{v}}{\cbr{v_a = \ol{v}_a}} \equiv
\rfun{H^\text{v}}{q, p; \cbr{\rfun{\ol{v}^a}{q, p_a; \cbr{v_\alpha}}, 
v_\alpha}}.
\end{equation}

\emph{Subspace of primary constraints}
\begin{equation}
\Gamma_\text{P} = \cbr{ \rbr{q, p}\, |\, \rfun{\Phi_\alpha}{q, p} = 0, 
\forall \alpha}
\end{equation}

Since
\begin{equation}
\frpa{H^\text{t}}{v_\alpha} =
\fat{\frpa{H^\text{v}}{v_\alpha}}{\cbr{v_a = \ol{v}_a}} = \Phi_\alpha
\equiv p_\alpha - \rfun{\ol{p}_\alpha}{q, \cbr{p_a}},
\end{equation}
$H^\text{t}$ is linear in $v_\alpha$. One writes
\begin{equation}
\rfun{H^\text{t}}{q, \cbr{p_a}; \cbr{p_\alpha}, \cbr{v_\alpha}} = \rfun{H}{q, 
\cbr{p_a}} + \sum_\alpha v_\alpha \Phi_\alpha,
\end{equation}
where $H^\text{c}$ is the \emph{canonical Hamiltonian} or simply 
\emph{Hamiltonian}.

\paragraph{Proposition}
$H^\text{c}$ is independent of $\cbr{p_\alpha}$.

\paragraph{Proposition}
Canonical equations with primary constraints
\begin{align}
\dot{q}_i &= \sbr{q_i, H}_\text{P} + \sum_\beta v_\beta 
\sbr{q_i, \phi_\beta}_\text{P},
\label{eq:q-i-primary}\\
\dot{p}_i &= \sbr{p_i, H}_\text{P} + \sum_\beta v_\beta 
\sbr{p_i, \phi_\beta}_\text{P}, \\
\rfun{\Phi_\alpha}{q, p} &= 0,
\end{align}
where $v_\beta$'s are undetermined. Note that \cref{eq:q-i-primary} for $i = 
\alpha$ holds identically: $\dot{q}_\alpha = \dot{q}_\alpha$.

Weak equality: $f_1 \approx f_2$ iff $\fat{f_1}{\Gamma_\text{P}} = 
\fat{f_2}{\Gamma_\text{P}}$.

\paragraph{Proposition} if $f$ and $g$ are two functions over the phase space 
$\Gamma$, and $f \approx h$, then
\begin{align}
\frpa{}{q_i} \rbr{f-\sum_\beta \phi_\beta \frpa{f}{p_\beta}} &\approx 
\frpa{}{q_i} \rbr{h-\sum_\beta \phi_\beta \frpa{h}{p_\beta}}, \\
\frpa{}{p_i} \rbr{f-\sum_\beta \phi_\beta \frpa{f}{p_\beta}} &\approx 
\frpa{}{p_i} \rbr{h-\sum_\beta \phi_\beta \frpa{h}{p_\beta}}.
\end{align}

\paragraph{Corollary}
$\forall H_1 \approx H$,
\begin{equation}
\dot{q}_i \approx \sbr{q_i, H}_\text{P},\qquad
\dot{p}_i \approx \sbr{p_i, H}_\text{P}.
\end{equation}

Primary and second constraints $\phi^{(1,)}_\mu$, $\phi^{(2,)}_\omega$; first 
and second class constraints $\phi^{(,1)}_u$, $\phi^{(,2)}_w$.



\section{Examples}

%\begin{equation}
%L^\text{v} = \frac{1}{2} \sum_{i,j}\rfun{W_{ij}}{q} v_i v_j + \sum_i 
%\rfun{\eta_i}{q} 
%v_i - \rfun{V}{q}.
%\end{equation}

%\begin{equation}
%p_i = \frpa{L^\text{v}}{v_i} = \sum_{i,j} W_{ij} v_j + \eta_i.
%\end{equation}

%Let
%\begin{align}
%\sum_j W_{ij} e_j^{(a)} &= \lambda^{(a)} e_i \neq 0, \\
%\sum_j W_{ij} e_j^{(\alpha)} &= 0.
%\end{align}

\subsection{Toy examples}

\subsubsection*{Example 0}
\cite[sec.\ 1.2]{Gitman1990}
\begin{equation}
L = \frac{1}{2}\rbr{\dot{x}-y}^2
\end{equation}


\subsubsection*{Example 1}
\begin{equation}
L = \frac{1}{2} \dot{x}^2 + \dot{x} y - \frac{1}{2}\rbr{x-y}^2.
\end{equation}

One has
\begin{equation}
L^\text{v} = \frac{1}{2} v_x^2 + v_x y - \frac{1}{2} \rbr{x-y}^2,
\end{equation}
so that
\begin{equation}
p_x = \frpa{L^\text{v}}{v_x} = v_x + y, \qquad p_y = 0,
\end{equation}
thus
\begin{equation}
\ol{v}_x = p_x - y.
\end{equation}
So that $v_y$ is the primary inexpressible velocity.

The Hamiltonian with velocity reads
\begin{equation}
\rfun{H^\text{v}}{q, p; v} = v_x p_x + v_y p_y - \frac{1}{2} v_x^2 - v_x y 
+ \frac{1}{2}\rbr{x-y}^2,
\end{equation}
whilst the total Hamiltonian is
\begin{equation}
\rfun{H^\text{t}}{q, p; \ol{v}_x, v_y} = \frac{1}{2}\rbr{p_x - y}^2 + 
\frac{1}{2} \rbr{x-y}^2 + v_y p_y.
\end{equation}

\subsubsection*{Example 2}

\begin{equation}
L = \frac{1}{2}\dot{x}^2 + \dot{x} y + \frac{1}{2}\rbr{x-y}^2
\end{equation}

Primary constraint
\begin{equation}
p_y = 0;
\end{equation}
total Hamiltonian
\begin{equation}
H^\text{t} = \frac{1}{2}p_x^2 - p_x y - \frac{1}{2} x^2 + xy + v_y p_y.
\end{equation}

\subsection*{Example 3}

\begin{equation}
L = \frac{1}{2} \rbr{\dot{q}_2 - \ee^{q_1}}^2 + \frac{1}{2} \rbr{\dot{q}_3 - 
q_2}^2.
\end{equation}



\subsection{Parametrised systems}

\subsubsection*{Non-relativistic point particle}

\cite[sec.\ 3.1.1]{Kiefer2012}
\begin{equation}
\sfun{S}{\rfun{q}{t}} \coloneqq \int_{t_1}^{t_2}\dd t\,\rfun{L}{q, \frde{q}{t}}
\end{equation}



\subsubsection*{Relativistic charged point particle}

\cite[sec.\ 16]{Landau1975},
\cite[sec.\ 3.1.2]{Kiefer2012}
\begin{equation}
S \coloneqq \int -m\,\dd s + e \rfun{A_\mu}{x} \,\dd x^\mu \eqqcolon \int\dd 
\tau\, L,\\
\label{eq:point-charged-action}
\end{equation}
where the Lagrangian reads
\begin{equation}
L = -m \sqrt{-\eta_{\mu\nu} \dot{x}^\mu \dot{x}^\nu } + q \dot{x}^\mu 
\rfun{A_\mu}{x}.
\end{equation}

\begin{equation}
M_{\mu\nu} \coloneqq \frpa{^2 L^\text{v}}{v^\mu\,\partial v^\nu} = 
m\frac{-\eta_{\mu\nu}\eta_{\alpha\beta} + \eta_{\mu\alpha}\eta_{\nu\beta}}% 
{\rbr{-\eta_{\rho\sigma}v^\rho v^\sigma}^{3/2}} v^\alpha v^\beta,
\end{equation}
which has one and only one eigenvector with null eigenvalue
\begin{equation}
v^\mu M_{\mu\nu} = 0.
\end{equation}

Momenta
\begin{equation}
p_\mu = \frpa{L^\text{v}}{v^\mu} = 
\frac{m\eta_{\mu\nu}v^\nu}{\sqrt{-\eta_{\rho\sigma}v^\rho v^\sigma}} + q A_\mu.
\label{eq:point-charged-pvrel}
\end{equation}
If one chooses $v^0$ to be the primary inexpressible velocity, then eliminating
$p_0$ in \cref{eq:point-charged-pvrel} yields
\begin{equation}
v^i = \frac{\xi \eta^{ij} \rbr{p_j - q A_j} v^0}{\sqrt{m^2 + \eta^{kl}
\rbr{p_k - q A_k}\rbr{p_l - q A_l}}},
\label{eq:point-charged-pvrel0}
\end{equation}
where $\xi = \sgn v^0$. In the following $\xi = +1$ will be chosen.

Inserting \cref{eq:point-charged-pvrel0} into the Hamiltonian with velocity
\begin{equation}
H^\text{v} = v^\mu p_\mu - L^\text{v} = m\sqrt{-\eta_{\mu\nu}v^\mu v^\nu} + 
v^\mu\rbr{p_\mu - q \rfun{A_\mu}{x}},
\end{equation}
one obtains the total Hamiltonian
\begin{equation}
H^\text{t} = v^0 \rbr{p_0 - q A_0 + \sqrt{m^2 + \eta^{kl}
\rbr{p_k - q A_k}\rbr{p_l - q A_l}}},
\end{equation}
where only a primary constraint survives, which is obviously a first-class 
constraint
\begin{equation}
\phi^{(1,1)} = p_0 - q A_0 + \sqrt{m^2 + \eta^{kl}
\rbr{p_k - q A_k}\rbr{p_l - q A_l}},
\end{equation}
and the canonical Hamiltonian vanishes
\begin{equation}
H^\text{c} = 0.
\end{equation}

To compare, note in the non-covariant formalism (\cite[sec.\ 8]{Landau1975})
\begin{equation}
S = \int \dd t\,L,\qquad L = -m\sqrt{1-\dot{\vec{x}}^2} - q \phi +
q \dot{\vec{x}} \cdot \vec{A},
\end{equation}
the system is regular, and the canonical Hamiltonian reads
\begin{equation}
H^\text{c} = \sqrt{m^2 + \rbr{\vec{p}-q\vec{A}}^2} + q\phi,
\end{equation}
which corresponds to setting $\phi^{(1,1)} = 0$, $p_0 \to -H^\text{c}$ 
($p_\mu = \rbr{-E, \vec{p}}$), and noting $A_\mu = \rbr{-\phi, \vec{A}}$.

\subsubsection*{Relativistic point particle with einbein}

\cite[sec.\ 2.1]{Blumenhagen2013}
\begin{equation}
L \coloneqq \frac{1}{2} \rbr{e^{-1}\eta_{\mu\nu}\dot{x}^\mu \dot{x}^\nu - m^2 e}
\label{eq:point-aux-lagrangian}
\end{equation}

\begin{equation}
p_\mu = \frpa{L^\text{v}}{v^\mu} = e^{-1}\eta_{\mu\nu}v^\nu, \qquad
p_e = 0.
\end{equation}
Choosing $v^e$ to be the primary inexpressible velocity, one has
\begin{equation}
v^\mu = e\eta^{\mu\nu}p_\nu.
\end{equation}

Hamiltonian with velocity
\begin{equation}
H^\text{v} = v^\mu p_\mu + v^e p_e + \frac{1}{2} \rbr{-e^{-1} \eta_{\mu\nu} 
v^\mu v^\nu + m^2 e};
\end{equation}
total Hamiltonian
\begin{equation}
H^\text{t} = \frac{e}{2} \rbr{\eta^{\mu\nu}p_\mu p_\nu + m^2} + v^e p_e;
\end{equation}
canonical Hamiltonian
\begin{equation}
H^\text{c} = \frac{e}{2} \rbr{\eta^{\mu\nu}p_\mu p_\nu + m^2}.
\end{equation}

The only primary constraint
\begin{equation}
\Phi^{(1,)} = p_e;
\end{equation}
its time evolution
\begin{align}
\sbr{\Phi^{(1,)}, H^\text{t}}_\text{P} &=
\sbr{p_e, e}_\text{P}\frac{1}{2}\rbr{\eta^{\mu\nu}p_\mu p_\nu + m^2}
\nonumber \\
&= -\frac{1}{2}\rbr{\eta^{\mu\nu}p_\mu p_\nu + m^2}.
\end{align}
Choose
\begin{equation}
\Phi^{(2,)} = \eta^{\mu\nu}p_\mu p_\nu + m^2,
\end{equation}
whose Possion bracket with $H^\text{t}$ vanishes; furthermore,
\begin{equation}
\sbr{\Phi^{(1,)},\Phi^{(2,)}}_\text{P} \equiv 0.
\end{equation}
Thus one ends up with two first-class constraints.



\subsubsection{Neutral scalar field}
\cite[sec.\ 3.3]{Kiefer2012}

\subsection{Maxwell--Proca theory}

\begin{equation}
\Ld = -\frac{1}{4} F_{\mu\nu} F^{\mu\nu} - \frac{1}{2} m^2 A_\mu A^\mu
+ A_\mu J^\mu,
\end{equation}
where $m > 0$ corresponds to the Proca theory \cite[sec.\ 2.3]{Gitman1990}, and 
$m = 0$ the Maxwell theory \cite[sec.\ 3.3.3]{Rothe2010}, \cite[sec.\ 
2.4]{Gitman1990}.

Lagrangian density with velocity
\begin{equation}
\Ld^\text{v} = \frac{1}{2} \rbr{V_i - \partial_i A_0}^2 - \frac{1}{4} F_{ij}^2 
+ \frac{m^2}{2} \rbr{A_0^2 - A_i^2} + A_0 J^0 + A_i J^i;
\end{equation}
momenta density
\begin{equation}
B^0 \coloneqq \frpa{\Ld^\text{v}}{V_0} = 0,\qquad
B^i \coloneqq \frpa{\Ld^\text{v}}{V_i} = V^i + \partial_i A^0;
\end{equation}
total and canonical Hamiltonian as well as primary constraint
\begin{align}
\mscrH^\text{t} &= \mscrH^\text{c} + V_0 \Phi^{(1,)},\\
\mscrH^\text{c} &= \frac{1}{2} B_i^2 + B_i \partial_i A_0 + \frac{1}{4} 
F_{ij}^2 + \frac{m^2}{2} \rbr{-A_0^2 + A_i^2} - A_0 J^0 - A_i J^i,\\
\Phi^{(1,)} &= B^0.
\end{align}

\begin{align}
\sbr{{\Phi^{(1,)}}_1,{\mscrH^\text{t}_2}}_\text{P} &=
{B^i}_1 {\partial_i}_2 \sbr{{B^0}_2, {A_0}_2}_\text{P} +
\frac{m^2}{2}\sbr{{B^0}_1, {A_0^2}_2}_\text{P} -
\sbr{{B^0}_1, {A_0}_2}_\text{P} {J^0}_2 \nonumber \\
&= \sbr{-{B^i}_2 {\partial_i}_2 - m^2 A_0 + {J^0}_2}_\text{P}
\rfun{\delta}{x_1 - x_2}.
\end{align}

\begin{equation}
\Phi^{(2,)} = \partial_i B^i - m^2 A_0 + J^0.
\end{equation}






%\subsection{Dirac field}

%\subsection{Gauge theories}

%\subsubsection{Spinor electrodynamics}

%\subsubsection{Yang--Mills theory}

%\subsubsection{Yang--Mills--Higgs theory}

\subsection{String theories}

\subsubsection*{Nambu--Gotō action}

Generalising the kinetic part of \eqref{eq:point-charged-action}, one has
\begin{equation}
S_\text{NG} \coloneqq -T \int_\Sigma \dd A
\eqqcolon -T \int_\Sigma\dd^2\sigma \Ld,
\end{equation}
where the Lagrangian density
\begin{equation}
\Ld = \sqrt{-\Gamma},\quad
\Gamma \coloneqq \det \Gamma_{\alpha\beta},\quad
\Gamma_{\alpha\beta} \coloneqq \frpa{X^\nu}{\sigma^\alpha} 
\frpa{X_\nu}{\sigma^\alpha}.
\end{equation}


Historically \cite{Nambu1970,Goto1971}; Reference e.g.\ 
\cite{Blumenhagen2013}
\cite[sec.\ 3.2]{Kiefer2012}

\subsubsection*{Polyakov action}

Generalising \eqref{eq:point-aux-lagrangian}
\begin{equation}
\sfun{S_\text{P}}{X^\mu, h_{\alpha\beta}} = -\frac{T}{2}\int_\Sigma \Ld,
\end{equation}
where
\begin{equation}
\Ld \coloneqq \sqrt{-h} h^{\alpha\beta}\Gamma_{\alpha\beta}.
\end{equation}



Historically \cite{Brink1976,Deser1976,Polyakov1981};
Reference
\cite[sec.\ 3.2]{Kiefer2012}







\subsection{Gravitation theories}

\subsubsection*{Closed Friedmann universe}
This part adapts \cite[sec.\ 8.1.2]{Kiefer2012}.

The total action reads
\begin{equation}
S \coloneqq S_\text{EG} + S_\phi,
\end{equation}
where $S_\text{EG}$ follows \eqref{eq:action-einstein-gravity}, and
\begin{equation}
S_\phi \coloneqq \int_\mscrM\dd^4 x\, \sqrt{-g}\,
\rbr{-\frac{1}{2} g^{\mu\nu} \rbr{\nabla_\mu\phi} \rbr{\nabla_\nu\phi}
-m^2\phi^2}.
\end{equation}

Adapting
\begin{equation}
\dd s^2 = -\rfun{N^2}{t}\,\dd t^2 + \rfun{a^2}{t}\,\dd\Omega_3^2,
\end{equation}
where
\begin{equation}
\d\Omega_3^2 = \dd\chi^2+\sin^2\chi\,\rbr{\dd\theta^2+\sin^2\theta\,\dd\phi^2}.
\end{equation}
One has
\begin{equation}
\sqrt{-g} = N a^3 \sin^2\chi\,\sin\theta,\qquad
\sqrt{h} = a^3\sin^2\chi\,\sin\theta;
\end{equation}
whereas
\begin{equation}
R = \frac{6}{N^2}\rbr{-\frac{\dot{N}\dot{a}}{Na} + \frac{\ddot{a}}{a} + 
\rbr{\frac{\dot{a}}{a}}^2} + \frac{6}{a^2},\qquad
K = \frac{3\dot{a}}{Na}.
\end{equation}

\begin{equation}
S_\text{EG} = \frac{A_3}{16\pp\nG} \rbr{\int_{t_1}^{t_2} \dd t
Na^3\rbr{R - 2\Lambda} - \sbr{\frac{6\dot{a}a^2}{N}}_{t_1}^{t_2}},
\end{equation}
where
\begin{equation}
A_3 = \int \sin^2\chi\,\sin\theta\,\dd\chi\,\dd\theta\,\dd\phi = 2\pp^2.
\end{equation}
The term proportional to $\ddot{a}/a$ in the integrand can be integrated by
parts
\begin{equation}
\int_{t_1}^{t_2} \dd t\,Na^3 \frac{6}{N^2} \frac{\ddot{a}}{a}
= 6\rbr{\sbr{\frac{\dot{a}a^2}{N}}_{t_1}^{t_2} - \int_{t_1}^{t_2}\dd t
\,\dot{a}\frde{}{t}\frac{a^2}{N^2}},
\end{equation}
in which the first term cancels the Gibbons--Hawking--York term. One has
\begin{equation}
S_\text{EG} = \frac{3\pp}{4\nG}\int_{t_1}^{t_2} \dd t\,
\rbr{-\frac{a}{N}\dot{a}^2 + N a - \frac{\Lambda}{3}N a^3}.
\end{equation}

The matter part of the action reads
\begin{equation}
S_\phi = \pp^2 \int_{t_1}^{t_2}\dd t\,
a^3 \rbr{\frac{1}{N}\dot{\phi}^2 - m^2N\phi^2}.
\end{equation}

Lagrangian with velocity
\begin{equation}
L^\text{v} = \frac{3\pp}{4\nG}\rbr{-\frac{a}{N} {v^a}^2 + Na - 
\frac{\Lambda}{3} Na^3} + \pp^2 a^3 \rbr{\frac{1}{N}{v^\phi}^2 - m^2 N 
\phi^2}.
\end{equation}
Canonical momenta
\begin{equation}
p_N \coloneqq \frpa{L^\text{v}}{v^N} = 0,\quad
p_a \coloneqq \frpa{L^\text{v}}{v^a} = -\frac{3\pp}{2\nG} \frac{a}{N} v^a,\quad
p_\phi \coloneqq \frpa{L^\text{v}}{v^\phi} = 2\pp^2 \frac{a^3}{N} v^\phi.
\end{equation}
Choosing $v^N$ to be the primary inexpressible velocity, one obtains the total 
and canonical Hamiltonians
\begin{align}
H^\text{t} &= H^\text{c} + v^N \Phi, \\
H^\text{c} &= -N H_\perp,
\end{align}
where
\begin{align}
H_\perp &\coloneqq \frac{\nG}{3\pp} \frac{p_a^2}{a}  - \frac{1}{4\pp^2} 
\frac{p_\phi^2}{a^3}  - \frac{3\pp}{4\nG}\rbr{\frac{\Lambda}{3}a^2 - 1}a
- \pp^2 m^2 a^3 \phi^2, \\
\Phi &= p_N
\end{align}
are the \emph{Hamiltonian constraint} and the primary constraint, respectively.

Evaluating the time evolution of $\Phi$ yields
\begin{equation}
\sbr{\Phi,H^t}_\text{P} = H_\perp,
\end{equation}
so that the Hamiltonian constraint is indeed a constraint. There is no further 
constraint, and $\sbr{\Phi, H_\perp}_\text{P}$ vanishes identically. Therefore 
there exists and only exists two second-class constraints.





\subsubsection{Einstein--Hilbert action}

\begin{equation}
S_\text{EG} = S_\text{EH} + S_\text{GHY},
\label{eq:action-einstein-gravity}
\end{equation}
\begin{equation}
S_\text{EH} = \frac{1}{16\pp\nG}\int_\mscrM\dd^4 x\,\sqrt{-g} 
\rbr{R-2\mitLambda},
\end{equation}
and
\begin{equation}
S_\text{GHY} = -\frac{1}{8\pp\nG}\int_{\partial\mscrM}\dd^3 x\,\sqrt{h} K,
\end{equation}
which is named after \cite{Gibbons1977,York1972} but actually already mentioned 
in \cite{Einstein1916}. See \cite{Dyer2009} for a brief review.


\printbibliography

\end{document}
