\documentclass[a4paper,11pt]{article}
%\documentclass[a4paper,10pt]{scrartcl}

\input{../preambles/preamble}
\input{../preambles/unicode}

\setmainlanguage{english}
\setotherlanguages{german,greek,russian}

\input{../preambles/math-single}
\input{../preambles/math-brac}
\input{../preambles/math-thm}
\input{../preambles/phys-chem}

\setromanfont[Mapping=tex-text]{Linux Libertine O}
% \setsansfont[Mapping=tex-text]{DejaVu Sans}
% \setmonofont[Mapping=tex-text]{DejaVu Sans Mono}

\usepackage[style=authoryear-icomp,
			backend=biber]{biblatex}
\addbibresource{../singular-dynamics.bib}

\title{Find the finite transformation}
\author{Yi-Fan Wang (王\ 一帆)}

\begin{document}
\maketitle

Given a lagrangian of a relativistic particle with an auxiliary einbein
\begin{equation}
L \coloneqq \frac{1}{2} \rbr{\frac{1}{\rfun{e}{t}}
\eta_{\mu\nu}\dot{x}^\mu \dot{x}^\nu - m^2 \rfun{e}{t}},
\label{eq:point-aux-lagrangian}
\end{equation}
where $\eta_{\mu\nu} \coloneqq \rfun{\diag}{-, +, +, \ldots}$, which is adapted 
from \cite[sec.\ 2.1]{Blumenhagen2013}. This is classically equivalent to
\begin{equation}
L = -m \sqrt{-\eta_{\mu\nu} \dot{x}^\mu \dot{x}^\nu }
\end{equation}
for $m>0$ (in the sense that the equations of motion for $x^\mu$'s are the 
same, when $\rfun{e}{t}$ eliminated), but also works for $m=0$.

One can use method in 
\cite[sec.\ 2]{Rothe2010} to \emph{solve for} the only gauge 
symmetry, whose \emph{infinitesimal} form reads
\begin{align}
\dva x^\mu &= \dot{x}^\mu \epsilon, \\
\dva e &= \dot{e} \epsilon + e \dot{\epsilon},
\end{align}
where $\epsilon = \rfun{\epsilon}{t}$ parametrises the transformation. The 
question is, how to solve the \emph{finite} transformation?

It is important to solve the transform, instead of guessing an answer, because 
there are more complicated scenarios where intuition is not enough.

\printbibliography

\end{document}
