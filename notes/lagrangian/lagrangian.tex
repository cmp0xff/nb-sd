\documentclass[a4paper,11pt]{article}
%\documentclass[a4paper,11pt]{scrartcl}



\input{../preambles/preamble}
\input{../preambles/unicode}

\setmainlanguage{english}
\setotherlanguages{german,greek,russian}

\input{../preambles/math-single}
\input{../preambles/math-brac}
\input{../preambles/math-thm}
\input{../preambles/phys-chem}

\setromanfont[Mapping=tex-text]{Linux Libertine O}
% \setsansfont[Mapping=tex-text]{DejaVu Sans}
% \setmonofont[Mapping=tex-text]{DejaVu Sans Mono}

\usepackage[style=authoryear-icomp,
			backend=biber]{biblatex}
\addbibresource{../singular-dynamics.bib}

\title{Notes on Lagrangian Singular Dynamics}
\author{Yi-Fan Wang (王\ 一帆)}
%\date{}

\begin{document}
\maketitle

\section{Classical formalism}

Lagrangian with velocity
\begin{equation}
L^\text{v} \coloneqq \fat{L}{\dot{q} = v}
\end{equation}
Equations of motion
\begin{equation}
\sum_j M_{ij}\dot{v}_j = K^\text{v}_i,\quad
\dot{q}_i = v_i.
\end{equation}
where
\begin{equation}
\rfun{M_{ij}}{q,v} \coloneqq \frpa{^2 L}{v_i\,\partial v_j}.
\end{equation}

% Adding
% \begin{equation}
% p_i \coloneqq \frpa{L^\text{v}}{v_i}.% \eqqcolon \rfun{\ol{p}_i}{q,v}.
% \end{equation}
% Variation of
% \begin{equation}
% \sfun{S}{q, p; v} \coloneqq \int\dd t\,\sbr{L^\text{v} + \sum_i 
% p_i\rbr{\dot{q}_i - v_i}}.
% \end{equation}
% gives the \emph{extended Euler--Lagrange equations}
% \begin{equation}
% \dot{q}_i = v_i,\quad
% \dot{p}_i = \frpa{L^\text{v}}{q_i},\quad
% p_i = \frpa{L^\text{v}}{v_i}.
% \end{equation}
% 
% Extended Hamiltonian
% \begin{equation}
% \rfun{H^\text{v}}{q, p; v} \coloneqq \sum_i p_i v_i - L^\text{v}.
% \end{equation}
% Identities
% \begin{equation}
% \frpa{H^\text{v}}{q_i} \equiv - \frpa{L^\text{v}}{q_i},\quad
% \frpa{H^\text{v}}{p_i} \equiv v_i,\quad
% \frpa{H^\text{v}}{v_i} \equiv p_i - \frpa{L^\text{v}}{v_i}.
% \end{equation}
% Variation of
% \begin{equation}
% \sfun{S}{q, p; v} \coloneqq \int\dd t\,\sbr{\sum_i 
% p_i \dot{q}_i - H^\text{v}}
% \end{equation}
% gives the \emph{extended canonical equations}
% \begin{equation}
% \dot{q}_i = \sbr{q_i, H^\text{v}}_\text{P},\quad
% \dot{p}_i = \sbr{p_i, H^\text{v}}_\text{P},\quad
% \frpa{H^\text{v}}{v_i} = 0,
% \end{equation}
% where the \emph{Poisson bracket} is defined as
% \begin{equation}
% \sbr{f^\text{v}, g^\text{v}}_\text{P} \coloneqq 
% \sum_i\rbr{\frpa{f^\text{v}}{q_i}\frpa{g^\text{v}}{p_i} -
% \frpa{f^\text{v}}{p_i}\frpa{g^\text{v}}{q_i}}.
% \end{equation}
% 
% $v_a = \rfun{\ol{v}_a}{q,p;\cbr{v_\alpha}}$ can be solved, $a = 1, 2, \ldots, 
% r_M$; $v_\alpha$ 
% cannot be solved, $\alpha = r_M + 1, \ldots, n$, where $r_M = \rank M$.
% 
% (need to show $v_a = \rfun{\ol{v}_a}{q,p_a}$)
% 
% \emph{Primary constraints in the standard form}
% \begin{equation}
% \rfun{\Phi_\alpha}{q, p} \coloneqq
% \fat{\frpa{H^\text{v}}{v_\alpha}}{\cbr{v_\alpha = \ol{v}_\alpha}} \equiv
% p_\alpha - \rfun{\ol{p}_\alpha}{q, \cbr{p_a}},
% \end{equation}
% where
% \begin{equation}
% \rfun{\ol{p}_\alpha}{q, \cbr{p_a}} 
% \coloneqq \fat{\frpa{L^\text{v}}{v_\alpha}}{\cbr{v_a = \ol{v}_a}}.
% \end{equation}
% 
% 
% \emph{Total Hamiltonian}
% \begin{equation}
% H^\text{t} \coloneqq \fat{H^\text{v}}{\cbr{v_a = \ol{v}_a}} \equiv
% \rfun{H^\text{v}}{q, p; \cbr{\rfun{\ol{v}^a}{q, p_a; \cbr{v_\alpha}}, 
% v_\alpha}}.
% \end{equation}
% 
% \emph{Subspace of primary constraints}
% \begin{equation}
% \Gamma_\text{P} = \cbr{ \rbr{q, p}\, |\, \rfun{\Phi_\alpha}{q, p} = 0, 
% \forall \alpha}
% \end{equation}
% 
% Since
% \begin{equation}
% \frpa{H^\text{t}}{v_\alpha} =
% \fat{\frpa{H^\text{v}}{v_\alpha}}{\cbr{v_a = \ol{v}_a}} = \Phi_\alpha
% \equiv p_\alpha - \rfun{\ol{p}_\alpha}{q, \cbr{p_a}},
% \end{equation}
% $H^\text{t}$ is linear in $v_\alpha$. One writes
% \begin{equation}
% \rfun{H^\text{t}}{q, \cbr{p_a}; \cbr{p_\alpha}, \cbr{v_\alpha}} = \rfun{H}{q, 
% \cbr{p_a}} + \sum_\alpha v_\alpha \Phi_\alpha,
% \end{equation}
% where $H$ is the \emph{canonical Hamiltonian} or simply \emph{Hamiltonian}.
% 
% \paragraph{Proposition}
% $H$ is independent of $\cbr{p_\alpha}$.
% 
% \paragraph{Proposition}
% Canonical equations with primary constraints
% \begin{align}
% \dot{q}_i &= \sbr{q_i, H}_\text{P} + \sum_\beta v_\beta 
% \sbr{q_i, \phi_\beta}_\text{P},
% \label{eq:q-i-primary}\\
% \dot{p}_i &= \sbr{p_i, H}_\text{P} + \sum_\beta v_\beta 
% \sbr{p_i, \phi_\beta}_\text{P}, \\
% \rfun{\Phi_\alpha}{q, p} &= 0,
% \end{align}
% where $v_\beta$'s are undetermined. Note that \cref{eq:q-i-primary} for $i = 
% \alpha$ holds identically: $\dot{q}_\alpha = \dot{q}_\alpha$.
% 
% Weak equality: $f_1 \approx f_2$ iff $\fat{f_1}{\Gamma_\text{P}} = 
% \fat{f_2}{\Gamma_\text{P}}$.
% 
% \paragraph{Proposition} if $f$ and $g$ are two functions over the phase space 
% $\Gamma$, and $f \approx h$, then
% \begin{align}
% \frpa{}{q_i} \rbr{f-\sum_\beta \phi_\beta \frpa{f}{p_\beta}} &\approx 
% \frpa{}{q_i} \rbr{h-\sum_\beta \phi_\beta \frpa{h}{p_\beta}}, \\
% \frpa{}{p_i} \rbr{f-\sum_\beta \phi_\beta \frpa{f}{p_\beta}} &\approx 
% \frpa{}{p_i} \rbr{h-\sum_\beta \phi_\beta \frpa{h}{p_\beta}}.
% \end{align}
% 
% \paragraph{Corollary}
% $\forall H_1 \approx H$,
% \begin{equation}
% \dot{q}_i \approx \sbr{q_i, H}_\text{P},\qquad
% \dot{p}_i \approx \sbr{p_i, H}_\text{P}.
% \end{equation}
% 
% Primary and second constraints $\phi^{(1,)}_\mu$, $\phi^{(2,)}_\omega$; first 
% and second class constraints $\phi^{(,1)}_u$, $\phi^{(,2)}_w$.



\section{Examples}

%\begin{equation}
%L^\text{v} = \frac{1}{2} \sum_{i,j}\rfun{W_{ij}}{q} v_i v_j + \sum_i 
%\rfun{\eta_i}{q} 
%v_i - \rfun{V}{q}.
%\end{equation}

%\begin{equation}
%p_i = \frpa{L^\text{v}}{v_i} = \sum_{i,j} W_{ij} v_j + \eta_i.
%\end{equation}

%Let
%\begin{align}
%\sum_j W_{ij} e_j^{(a)} &= \lambda^{(a)} e_i \neq 0, \\
%\sum_j W_{ij} e_j^{(\alpha)} &= 0.
%\end{align}

\subsection{Toy examples}

\subsubsection*{Example 0}
\cite[sec.\ 1.2]{Gitman1990}
\begin{equation}
L = \frac{1}{2}\rbr{\dot{x}-y}^2
\end{equation}


\subsubsection*{Example 1}
\begin{equation}
L = \frac{1}{2} \dot{x}^2 + \dot{x} y - \frac{1}{2}\rbr{x-y}^2.
\end{equation}

% One has
% \begin{equation}
% L^\text{v} = \frac{1}{2} v_x^2 + v_x y - \frac{1}{2} \rbr{x-y}^2,
% \end{equation}
% so that
% \begin{equation}
% p_x = \frpa{L^\text{v}}{v_x} = v_x + y, \qquad p_y = 0,
% \end{equation}
% thus
% \begin{equation}
% \ol{v}_x = p_x - y.
% \end{equation}
% So that $v_y$ is the primary inexpressible velocity.
% 
% The extended Hamiltonian reads
% \begin{equation}
% \rfun{H^\text{v}}{q, p; v} = v_x p_x + v_y p_y - \frac{1}{2} v_x^2 - v_x y 
% + \frac{1}{2}\rbr{x-y}^2,
% \end{equation}
% whilst the total Hamiltonian is
% \begin{equation}
% \rfun{H^\text{t}}{q, p; \ol{v}_x, v_y} = \frac{1}{2}\rbr{p_x - y}^2 + 
% \frac{1}{2} \rbr{x-y}^2 + v_y p_y.
% \end{equation}

\subsubsection*{Example 2}

\begin{equation}
L = \frac{1}{2}\dot{x}^2 + \dot{x} y + \frac{1}{2}\rbr{x-y}^2
\end{equation}

% Primary constraint
% \begin{equation}
% p_y = 0;
% \end{equation}
% total Hamiltonian
% \begin{equation}
% H^\text{t} = \frac{1}{2}p_x^2 - p_x y - \frac{1}{2} x^2 + xy + v_y p_y.
% \end{equation}

\subsection*{Example 3}

\begin{equation}
L = \frac{1}{2} \rbr{\dot{q}_2 - \ee^{q_1}}^2 + \frac{1}{2} \rbr{\dot{q}_3 - 
q_2}^2.
\end{equation}



\subsection{Parametrised systems}

\subsubsection*{Non-relativistic point particle}

\cite[sec.\ 3.1.1]{Kiefer2012}
\begin{equation}
\sfun{S}{\rfun{q}{t}} \coloneqq \int_{t_0}^{t_1}\dd t\,\rfun{L}{q, \frde{q}{t}}
\end{equation}



\subsubsection*{Relativistic charged point particle}

\cite[sec.\ 16]{Landau1975},
\cite[sec.\ 3.1.2]{Kiefer2012}
\begin{equation}
S \coloneqq \int_\gamma -m\,\dd s + e \rfun{A_\mu}{x} \,\dd x^\mu
\eqqcolon \int_{\tau_0}^{\tau_1} \dd \tau\, L,\\
\label{eq:point-charged-action}
\end{equation}
where the Lagrangian reads
\begin{equation}
L = -m \sqrt{-\eta_{\mu\nu} \dot{x}^\mu \dot{x}^\nu } + q \dot{x}^\mu 
\rfun{A_\mu}{x}.
\end{equation}

\begin{equation}
M_{\mu\nu} \coloneqq \frpa{^2 L}{\dot{x}^\mu\,\partial \dot{x}^\nu} = 
m\frac{-\eta_{\mu\nu}\eta_{\alpha\beta} + \eta_{\mu\alpha}\eta_{\nu\beta}}% 
{\rbr{-\eta_{\rho\sigma}\dot{x}^\rho \dot{x}^\sigma}^{3/2}} \dot{x}^\alpha 
\dot{x}^\beta,
\end{equation}
which has one and only one zero eigenvector
\begin{equation}
\dot{x}^\mu M_{\mu\nu} = 0.
\end{equation}

Euler--Lagrange derivatives
\begin{equation}
E_\mu = \rbr{\frpa{}{x^\mu}-\frde{}{\tau}\frpa{}{\dot{x}^\mu}} L
\equiv K_\mu - M_{\mu\nu}\ddot{x}^\nu,
\label{eq:point-charged-eld}
\end{equation}
where
\begin{equation}
K_\mu \coloneqq -q F_{\mu\nu} \dot{x}^\nu,
\end{equation}
and
\begin{equation}
F_{\mu\nu} \coloneqq \partial_\mu A_\nu - \partial_\nu A_\mu.
\end{equation}

Contracting the zero eigenvector with \cref{eq:point-charged-eld} yields
\begin{equation}
\dot{x}^\mu E_\mu = \dot{x}^\mu K_\mu - \dot{x}^\mu M_{\mu\nu} \ddot{x}^\nu
\equiv 0,
\end{equation}
so that it generates a gauge identity, and no further constraint exists. Thus 
the system has a symmetry
\begin{equation}
\dva x^\mu = \dot{x}^\mu \dva \lambda.
\end{equation}

\subsubsection*{Relativistic point particle with einbein}

\cite[sec.\ 2.1]{Blumenhagen2013}
\begin{equation}
L \coloneqq \frac{1}{2} \rbr{e^{-1}\eta_{\mu\nu}\dot{x}^\mu \dot{x}^\nu - m^2 e}
\label{eq:point-aux-lagrangian}
\end{equation}

Euler--Lagrange derivatives
\begin{align}
E_\mu &\coloneqq \rbr{\frpa{}{x^\mu}-\frde{}{\tau}\frpa{}{\dot{x}^\mu}}L 
= e^{-1}\eta_{\mu\nu} \rbr{\frac{\dot{e}}{e}\dot{x}^\nu - \ddot{x}^\nu}, \\
E_e &\coloneqq \rbr{\frpa{}{e}-\frde{}{\tau}\frpa{}{\dot{e}}}L
= -\frac{1}{2}\rbr{\frac{1}{e^2}\eta_{\mu\nu}\dot{x}^\mu\dot{x}^\nu + m^2},
\end{align}
and collectively $E^{(0)} = \begin{pmatrix} E_\mu & E_e 
\end{pmatrix}^\intercal$.

\begin{equation}
M^{(0)} \coloneqq M =
\begin{pmatrix}
e^{-1}\eta_{\mu\nu} & 0^\mu \\
0^\nu & 0
\end{pmatrix},
\end{equation}
so that the system is singular, with $w^{(0)} = \rbr{0^\mu; 1}$.

One can choose $u^{(0)} = \rbr{0^\mu; e^2}$, so that
\begin{align}
\phi^{(0)} &\coloneqq u^{(0)}\cdot E^{(0)} = e^2 E_e \\
&= -\frac{1}{2}\rbr{\eta_{\mu\nu}\dot{x}^\mu\dot{x}^\nu + m^2 e^2},
\end{align}
and thus
\begin{align}
E^{(1)}_1 &\coloneqq \dot{\phi}^{(0)} = e\rbr{2\dot{e} E_e + e\dot{E}_e} \\
&= - m^2 e\dot{e}-\eta_{\mu\nu}\dot{x}^\mu\ddot{x}^\nu.
\end{align}
Collectively, $E^{(1)} = \begin{pmatrix} \rbr{E^{(0)}}^\intercal & E^{(1)}_1
\end{pmatrix}^\intercal$.

Straightforwardly,
\begin{equation}
M^{(1)} =
\begin{pmatrix}
e^{-1}\eta_{\mu\nu} & 0^\mu \\
0^\nu & 0 \\
\eta_{\mu\nu}\dot{x}^\mu & 0
\end{pmatrix},
\end{equation}
and the new zero eigenvector $w^{(1)} = \rbr{e\dot{x}^\mu; 0, -1}$.

One finds that
\begin{align}
w^{(1)} \cdot E^{(1)} &= e\dot{x}^\mu E_\mu - E^{(1)}_1 =
e\rbr{\dot{x}^\mu E_\mu - 2 \dot{e} E_e - e\dot{E}_e} \\
&= \eta_{\mu\nu} \frac{\dot{e}}{e} \dot{x}^\mu \dot{x}^\nu  + m^2 e \dot{e}
= -2e \dot{e} E_e,
\end{align}
so that a gauge identity
\begin{equation}
G \coloneqq \dot{x}^\mu E_\mu - e \dot{E}_e \equiv 0
\end{equation}
is obtained.

\begin{equation}
G\epsilon = E_\mu \dot{x}^\mu \epsilon + E_e \rbr{\dot{e} \epsilon + e 
\dot{\epsilon}} - \frde{}{\tau}\rbr{e E_e \epsilon},
\end{equation}
so that
\begin{align}
\dva x^\mu &= \dot{x}^\mu \epsilon, \\
\dva e &= \dot{e} \epsilon + e \dot{\epsilon}.
\end{align}






\subsubsection{Neutral scalar field}
\cite[sec.\ 3.3]{Kiefer2012}

\subsection{Maxwell--Proca theory}

\begin{equation}
\Ld = -\frac{1}{4} F_{\mu\nu} F^{\mu\nu} - \frac{1}{2}m^2 A_\mu A^\mu + A_\mu 
J^\mu
\end{equation}
where $m > 0$ corresponds to the Proca theory \cite[sec.\ 2.3]{Gitman1990}, and 
$m = 0$ the Maxwell theory \cite[sec.\ 3.3.3]{Rothe2010}, \cite[sec.\ 
2.4]{Gitman1990}.

\begin{equation}
\Ld \equiv \frac{1}{2}\rbr{-\eta^{\alpha\beta}\eta^{\mu\nu} + \eta^{\alpha\nu} 
\eta^{\beta^\mu}}\rbr{\partial_\mu A_\alpha}\rbr{\partial_\nu A_\beta}
- \frac{1}{2} m^2 \eta^{\alpha\beta} A_\alpha A_\beta + A_\alpha J^\alpha
\end{equation}


\begin{align}
E^\alpha &= \rbr{\frpa{}{A_\alpha} - \partial_\mu \frpa{}{\rbr{\partial_\mu 
A_\alpha}}}\Ld \nonumber \\
&= -m^2 A_\beta \eta^{\alpha\beta} + J^\alpha -\rbr{-\eta^{\alpha\beta} 
\eta^{\mu\nu} + \eta^{\alpha\nu} \eta^{\beta^\mu}}
\partial_\mu \partial_\nu A_\beta.
\end{align}



%\subsection{Dirac field}

%\subsection{Gauge theories}

%\subsubsection{Spinor electrodynamics}

%\subsubsection{Yang--Mills theory}

%\subsubsection{Yang--Mills--Higgs theory}

\subsection{String theories}

\subsubsection*{Nambu--Gotō action}

Generalising the kinetic part of \eqref{eq:point-charged-action}, one has
\begin{equation}
S_\text{NG} \coloneqq -T \int_\Sigma \dd A
\eqqcolon -T \int_\Sigma\dd^2\sigma \Ld,
\end{equation}
where the Lagrangian density
\begin{equation}
\Ld = \sqrt{-\Gamma},\quad
\Gamma \coloneqq \det \Gamma_{\alpha\beta},\quad
\Gamma_{\alpha\beta} \coloneqq \frpa{X^\nu}{\sigma^\alpha} 
\frpa{X_\nu}{\sigma^\alpha}.
\end{equation}


Historically \cite{Nambu1970,Goto1971}; Reference e.g.\ 
\cite{Blumenhagen2013}
\cite[sec.\ 3.2]{Kiefer2012}

\subsubsection*{Polyakov action}

Generalising \eqref{eq:point-aux-lagrangian}
\begin{equation}
\sfun{S_\text{P}}{X^\mu, h{\alpha\beta}} = -\frac{T}{2}\int_\Sigma \Ld,
\end{equation}
where
\begin{equation}
\Ld \coloneqq \sqrt{-h} h^{\alpha\beta}\Gamma_{\alpha\beta}.
\end{equation}



Historically \cite{Brink1976,Deser1976,Polyakov1981};
Reference
\cite[sec.\ 3.2]{Kiefer2012}


\begin{equation}
\sqrt{}
\end{equation}



\subsection{Gravitation theories}

\subsubsection*{Closed Friedmann universe}
This part adapts \cite[sec.\ 8.1.2]{Kiefer2012}.

The total action reads
\begin{equation}
S \coloneqq S_\text{EG} + S_\phi,
\end{equation}
where $S_\text{EG}$ follows \eqref{eq:action-einstein-gravity}, and
\begin{equation}
S_\phi \coloneqq \int_\mscrM\dd^4 x\, \sqrt{-g}\,
\rbr{-\frac{1}{2} g^{\mu\nu} \rbr{\nabla_\mu\phi} \rbr{\nabla_\nu\phi}
-m^2\phi^2}.
\end{equation}

Adapting
\begin{equation}
\dd s^2 = -\rfun{N^2}{t}\,\dd t^2 + \rfun{a^2}{t}\,\dd\Omega_3^2,
\end{equation}
where
\begin{equation}
\d\Omega_3^2 = \dd\chi^2+\sin^2\chi\,\rbr{\dd\theta^2+\sin^2\theta\,\dd\phi^2}.
\end{equation}
One has
\begin{equation}
\sqrt{-g} = N a^3 \sin^2\chi\,\sin\theta,\qquad
\sqrt{h} = a^3\sin^2\chi\,\sin\theta;
\end{equation}
whereas
\begin{equation}
R = \frac{6}{N^2}\rbr{-\frac{\dot{N}\dot{a}}{Na} + \frac{\ddot{a}}{a} + 
\rbr{\frac{\dot{a}}{a}}^2} + \frac{6}{a^2},\qquad
K = \frac{3\dot{a}}{Na}.
\end{equation}

\begin{equation}
S_\text{EG} = \frac{A_3}{16\pp\nG} \rbr{\int_{t_1}^{t_2} \dd t
Na^3\rbr{R - 2\Lambda} - \sbr{\frac{6\dot{a}a^2}{N}}_{t_1}^{t_2}},
\end{equation}
where
\begin{equation}
A_3 = \int \sin^2\chi\,\sin\theta\,\dd\chi\,\dd\theta\,\dd\phi = 2\pp^2.
\end{equation}
The term proportional to $\ddot{a}/a$ in the integrand can be integrated by
parts
\begin{equation}
\int_{t_1}^{t_2} \dd t\,Na^3 \frac{6}{N^2} \frac{\ddot{a}}{a}
= 6\rbr{\sbr{\frac{\dot{a}a^2}{N}}_{t_1}^{t_2} - \int_{t_1}^{t_2}\dd t
\,\dot{a}\frde{}{t}\frac{a^2}{N^2}},
\end{equation}
in which the first term cancels the Gibbons--Hawking--York term. One has
\begin{equation}
S_\text{EG} = \frac{3\pp}{4\nG}\int_{t_1}^{t_2} \dd t\,
\rbr{-\frac{a}{N}\dot{a}^2 + N a - \frac{\Lambda}{3}N a^3}.
\end{equation}

The matter part of the action reads
\begin{equation}
S_\phi = \pp^2 \int_{t_1}^{t_2}\dd t\,
a^3 \rbr{\frac{1}{N}\dot{\phi}^2 - m^2\phi^2}.
\end{equation}

One derives the Euler--Lagrange derivatives
\begin{align}
E_N &= \frac{3\pp}{4\nG}\rbr{\frac{a\dot{a}^2}{N^2}+a-\frac{\Lambda a^3}{3}}
-\pp^2 a^3 \frac{\dot{\phi}^2}{N^2} ,\\
E_a &= \frac{3\pp}{4\nG}\rbr{-\frac{\dot{a}^2}{N}+N-\Lambda Na^2
+\frac{2a\ddot{a}}{N} - \frac{2a\dot{a}\dot{N}}{N^2}}
+3\pp^2a^2\rbr{\frac{\dot{\phi}^2}{N} - m^2\phi^2},\\
E_\phi &= 2\pp^2 \rbr{-m^2 a^3 \phi - \frac{3 a^2 \dot{a}\dot{\phi}}{N}
-\frac{a^3\ddot{\phi}}{N} + \frac{a^3\dot{\phi}\dot{N}}{N^2}},
\end{align}
and the primary mass matrix reads
\begin{equation}
\mbfM^{(0)} = \begin{pmatrix}
0 & 0 & 0 \\ 0 & -\frac{3\pp}{2\nG}\frac{a}{N} & 0 \\
0 & 0 & 2\pp^2\frac{a^3}{N},
\end{pmatrix}
\end{equation}
so that the system is singular, with $w^{(0)} = \begin{pmatrix}
1 & 0 & 0\end{pmatrix} \eqqcolon u^{(0)}$.

Therefore, the only primary constraint
\begin{equation}
\phi^{(0)} \coloneqq u^{(0)}\cdot E^{(0)} = E_N,
\end{equation}
and
\begin{align}
E^{(1)} &\coloneqq \dot{\phi}^{(0)} = \dot{E}_N \\
&= \frac{3\pp}{4\nG}\dot{a} 
\rbr{\frac{\dot{a}^2}{N^2}+2\frac{a\ddot{a}}{N^2}
-\frac{2a\dot{a}\dot{N}}{N^3}+1-\Lambda a^2} - \pp^2 \frac{a^2\dot{\phi}}{N^2}
\rbr{3\dot{a}\dot{\phi}+2a\ddot{\phi}-2a\frac{\dot{N}}{N}}.
\end{align}
The secondary mass matrix
\begin{equation}
\mbfM^{(1)} = \begin{pmatrix}
0 & 0 & 0 \\ 0 & -\frac{3\pp}{2\nG}\frac{a}{N} & 0 \\
0 & 0 & 2\pp^2\frac{a^3}{N} \\
0 & -\frac{3\pp}{2\nG}\frac{a\dot{a}}{N^2} & 2\pp^2\frac{a^3\dot{\phi}}{N^2},
\end{pmatrix}
\end{equation}
and an additional left zero
\begin{equation}
w^{(1)} \coloneqq \begin{pmatrix}
0 & \frac{\dot{a}}{N} & \frac{\dot{\phi}}{N} & -1
\end{pmatrix}
\eqqcolon v^{(1)}
\end{equation}
is obtained, resulting in a gauge identity
\begin{equation}
0 \equiv G = \frac{\dot{a}}{N} E_a + \frac{\dot{\phi}}{N} E_\phi - \dot{E}_N,
\end{equation}
which terminates the algorithm. The gauge transformation reads
\begin{equation}
\dva N = \dot{\epsilon},\qquad \dva a = \frac{\dot{a}}{N}\epsilon,\qquad
\dva \phi = \frac{\dot{\phi}}{N}\epsilon.
\end{equation}







\subsubsection{Einstein gravity}

\begin{equation}
S_\text{EG} = S_\text{EH} + S_\text{GHY},
\label{eq:action-einstein-gravity}
\end{equation}
where the Einstein--Hilbert action
\begin{equation}
S_\text{EH} = \frac{1}{16\pp\nG}\int_\mscrM\dd^4 x\,\sqrt{-g}\, 
\rbr{R-2\mitLambda},
\end{equation}
and the Gibbons--Hawking--York action
\begin{equation}
S_\text{GHY} = -\frac{1}{8\pp\nG}\int_{\partial\mscrM}\dd^3 x\,\sqrt{h}\, K,
\end{equation}
which is named after \cite{Gibbons1977,York1972} but actually already
mentioned in \cite{Einstein1916}. See \cite{Dyer2009} for a brief review.



\printbibliography

\end{document}
